\documentclass{article}

\usepackage[french]{babel}
\usepackage[utf8]{inputenc}
\usepackage[T1]{fontenc}
\usepackage[a4paper]{geometry}
\usepackage{graphicx}
\usepackage{hyperref}
\hypersetup{pdftex,colorlinks=true,allcolors=black}
\usepackage{hypcap}
\usepackage{url}

\nocite{*}

\title{%
  PROSIT 2 Aller \\
  \large  Un conteneur dans les nuages}
\date{\today}

\begin{document}

\maketitle

\tableofcontents

\pagebreak

\section{Contexte}
Léo veut mettre en place le logiciel GLPI mais Laila ne peut pas installer de machine virtuelle sur son ordi, Marco propose d'utiliser HyperV ou Docker.

\section{Mots clés}
\begin{itemize}
  \item ITIL (Information Technology Infrastructure Library) : ITIL est un cadre de référence pour la gestion des services informatiques. Il propose des bonnes pratiques pour aligner les services IT sur les besoins des entreprises, en couvrant des domaines tels que la gestion des incidents, des changements et des niveaux de service.
  \item Docker : Plateforme permettant de créer, déployer et exécuter des applications dans des conteneurs. Les conteneurs isolent les applications et leurs dépendances, assurant une portabilité accrue entre les environnements de développement, de test et de production.
  \item Hyperviseur : Un hyperviseur est un logiciel permettant de créer et de gérer des machines virtuelles. Type 1 : Exécuté directement sur le matériel physique (ex. VMware ESXi, Microsoft Hyper-V) ; Type 2 : Fonctionne au-dessus d’un système d’exploitation hôte (ex. VirtualBox, VMware Workstation).
  \item PoC (Proof of Concept) : Une preuve de concept est un prototype ou une démonstration visant à prouver la faisabilité ou l’efficacité d’une idée, d’un produit ou d’une technologie avant un développement à grande échelle.
  \item ITSN (Information Technology Service Network) : Réseaux dédiés à la fourniture de services IT, souvent utilisés dans un contexte de gestion centralisée des services IT et de communication entre les équipes ou systèmes.
  \item VDI (Virtual Desktop Infrastructure) : VDI permet d’héberger des environnements de bureau virtuels sur un serveur centralisé. Les utilisateurs accèdent à leurs bureaux via des clients légers ou à distance, offrant flexibilité et sécurité.
  \item Baremetal : Serveur physique sans couche logicielle intermédiaire (comme un hyperviseur). Les applications ou systèmes d’exploitation sont installés directement sur le matériel.
  \item Cluster (Hyper-V) : Ensemble de serveurs interconnectés qui travaillent ensemble pour assurer la disponibilité des machines virtuelles. Il offre des fonctionnalités comme la migration en direct et la tolérance aux pannes.
  \item RDP (Remote Desktop Protocol) : Protocole de Microsoft permettant de se connecter à distance à un autre ordinateur via une interface graphique. Il est souvent utilisé pour administrer des serveurs ou des postes de travail à distance.
  \item SSH (Secure Shell) : Protocole sécurisé permettant l’accès à distance à un serveur ou à un autre système informatique. Il offre des fonctionnalités telles que le transfert de fichiers et l’exécution de commandes sur des systèmes distants.
  \item GLPI (Gestion Libre de Parc Informatique) : Application open-source de gestion des ressources IT. Elle propose des fonctionnalités pour gérer les inventaires, les tickets, la maintenance et les contrats.
  \item Help desk : Service de support informatique fournissant une assistance aux utilisateurs pour résoudre des problèmes techniques ou répondre à des questions.
  \item PHP (Hypertext Preprocessor) : Langage de programmation côté serveur largement utilisé pour développer des applications web. Il est notamment utilisé dans les systèmes de gestion de contenu comme WordPress et les applications personnalisées.
\end{itemize}


\section{Problématique}

Comment déployer une solution ITSN/GLPI qui améliore la gestion actuelle sous Excel en respectant les contraintes de virtualisation ?


\section{Contraintes}
\begin{itemize}
  \item Pas de VirtualBox
  \item GLPI:
  \subitem Serveur web
  \subitem Base de données
  \item Pas de droits admin
\end{itemize}

\section{Livrables}
\begin{itemize}
  \item GLPI déployé sur Docker/HyperV
  \item Schéma d'architecture
  \item Comparatif Docker/HyperV
  \item Plan de déploiement
\end{itemize}

\section{Généralisations}
\begin{itemize}
  \item Conteneurisation
  \item Bonnes pratiques Git
  \item Virtualisation
  \item Serveur web
\end{itemize}

\section{Pistes de solutions}
\begin{itemize}
  \item Comparatif :
  \subitem Consommation
  \subitem Facilité de déploiement
  \item Utiliser les outils proposés par le logiciel pour l'installer
  \item Suivre les recommandations d'installation du logiciel
\end{itemize}

\section{Plan d'action}
\begin{enumerate}
  \item Analyser besoins d'installation GLPI
  \item Réfléchir aux étapes du déploiement
  \item Déployer GLPI sur Docker et HyperV
  \item Comparer
  \item Faire un guide de déploiement sur Docker et HyperV
\end{enumerate}

\section{Réalisation}

\subsection{Mathéo}
Avec docker-compose: permet de structurer un "docker run" dans un fichier. 2 containers: 1 pour la bdd, 1 pour GLPI. Transferable entre n'importe quelle machine

\subsection{Grégoire et Thibault}
Hyperviseur type 1 (Proxmox).\\
Serveur Apache qui fait tourner GLPI.\\
Instructions du site web GLPI.\

\end{document}